% cleaned of unnecessary equation numbers 3 June 2000
\subsection{Design of batch chemical plants}%
\emph{Due to Subrahmanyam, Pekny, and Reklaitis \cite{subrahmanyam94}}%

\noindent(Multistage, mixed integer linear stochastic problem)\\
\noindent \url{/chem}\url{/chem.cor},\url{/chem.tim},\url{chem.sto}

\vspace{3mm}
\subsubsection{Description}
Subrahmanyam, Pekny, and Reklaitis \cite{subrahmanyam94} describe the design of a batch type chemical plant to produce products for which we do not know the future demand.  We present here only half of the problem given in  \cite{subrahmanyam94}, the ``Design SuperProblem.''  

We must decide how many plants to build, of what type, when to build them, and how to operate them.  Therefore the problem has some integer decision variables.  Let $n_{jt}$ be the number of new units of equipment type $j$ to come online in time stage $t$.  This must take only integer values.  The cumulative number of units of type $j$ at time $t$ is defined as
\begin{equation*}
%\label{CHEM:cumplants}
N_{jt}\assign \sum_{\tau = 1}^t n_{j\tau}, \hspace{3mm} \forall j,t.
\end{equation*}

The various plants can perform different tasks, which are indexed as $i=1,\ldots, I$.  Certain plants can perform more than one task.  Let $I_j^{\text{tasks}}$ be the set of tasks from $1$ to $I$ which can be performed by a plant of type $j$, and let $I_i^{\text{equip}}$ be the set of plant types from $1$ to $J$ which can perform task $i$.

We must decide which tasks to perform on which equipment during each time period.  Let $y_{ijt}$ be the number of times task $i$ is performed on equipment type $j$ during time stage $t$.  If $p_{ij}$ is the processing time for task $i$ with equipment type $j$, and if $H_t$ is the length of stage $t$, then
\begin{equation*}
%\label{CHEM:timelimit}
\sum_{i\in I_j^{\text{tasks}}} p_{ij} y_{ijt} \leq H_t N_{jt}, \hspace{3mm} \forall j,t.
\end{equation*}
This constraint enforces the fact that time is limited in each stage.  Something else which might limit the number of batches is the much feared operating expense budget.  Let $C^o_t$ be the total operating budget per plant for time $t$, and let $c^o_{ijt}$ be the operating expense incurred for using equipment type $j$ to perform task $i$ in stage $t$.  Then the operating expense constraint is
\begin{equation*}
\sum_j \sum_{i\in I_j^{\text{tasks}}} c^o_{ijt} y_{ijt}\leq C^o_t \sum_j N_{jt}, \hspace{3mm} \forall t.
\end{equation*}

The material balance on the system includes inventory, production, consumption, sales, and purchasing effects.  Let $B_{ijt}$ be the amount of task $i$ performed on plant type $j$ in stage $t$, measured in somewhat arbitrary reaction units.  If $f_{si}$ is the stoichiometric ratio representing mass of resource $s$ produced per unit of reaction $i$, then the amount of $s$ produced in stage $t$ is
\[
\sum_i \sum_{j\in I_i^{\text{equip}}} f_{si}B_{ijt}.
\]
Note that $f_{si}$ is negative if resource $s$ is consumed in task $i$.  The mass of resource $s$ in inventory at the end of stage $t$ is $A_{st}$, with maximum limit $A_{st}^{\text{max}}$.  The material balance constraints are then
\begin{equation*}
%\label{CHEM:matbal}
A_{st} = A_{s(t-1)} + \sum_i \sum_{j\in I_i^{\text{equip}}} f_{si}B_{ijt} - q^s_{st} + q^b_{st}, \hspace{3mm} \forall s,t,
\end{equation*}
and
\begin{equation*}
%\label{CHEM:maxinv}
A_{st} \leq A_{st}^{\text{max}}, \hspace{3mm} \forall s,t,
\end{equation*}
where $q^s_{st}$ is the mass of resource $s$ sold in stage $t$, and $q^b_{st}$ is that bought in stage $t$.

The relationship between $B_{ijt}$ and $y_{ijt}$ is
\begin{equation*}
%\label{CHEM:batch}
B_{ijt} \leq m_{ij} y_{ijt}, \hspace{3mm} \forall i,j,t.
\end{equation*}
Here, $m_{ij}$ is the capacity of equipment type $j$ to perform task $i$, measured in units of reaction per batch.

There are a couple ways to limit purchases.  One is to simply impose a limit, as in
\begin{equation*}
%\label{CHEM:purlim}
q^b_{st} \leq Q^b_{st}, \hspace{3mm} \forall s,t,
\end{equation*}
for some constant $Q^b_{st}$.  Another is to limit capital expenditures to not exceed a constant $MC_t$, as in
\begin{equation*}
\sum_{j}C_{jt}n_{jt} + \sum_s v^b_{st} q^b_{st} \leq MC_t, \hspace{3mm} \forall t.
\end{equation*}
The symbol $C_{jt}$ is the capital investment cost for a plant of type $j$ in stage $t$.  The term $v^b_{st}$ is the value of purchased resource $s$ in stage $t$ per unit mass.

One of the random variables in this problem is the demand, $\rand{Q^s_{skt}}$, for resource $s$ at time $t$.  The index $k$ determines the scenario.  The other random variable is $\rand{v^s_{skt}}$, the price per unit mass of resource $s$ sold at or below the demand level $\rand{Q^s_{skt}}$ in stage $t$.  We define a new random variable by $\rand{r_{skt}}\assign(\rand{Q^s_{skt}}, \rand{v^s_{skt}})$.

The recourse variables are $q^{s0}_{skt}$, the amount of $s$ sold in stage $t$ which does not exceed demand, and $q^{s+}_{skt}$, the amount which exceeds demand.  Essentially, $q^{s+}_{skt}$ is given away, rather than sold, as no credit toward profit may be taken for this quantity.  The recourse variables are limited by the equation
\begin{equation*}
%\label{CHEM:soldbal}
q^s_{st} = q^{s0}_{skt} + q^{s+}_{skt}, \hspace{3mm} \forall s,k,t,
\end{equation*}
and the inequality
\begin{equation*}
%\label{CHEM:solddemand}
q^{s0}_{skt} \leq \rand{Q^s_{skt}}, \hspace{3mm} \forall s,k,t.
\end{equation*}

In some industries, it is important that the demand be met exactly.  For such cases, define the variable
\begin{equation*}
x_{kt}\assign
	\begin{cases}
		1& \text{if for each $s$, $q^{s0}_{skt} = \rand{Q^s_{skt}}$},\\
		0& \text{otherwise}.
	\end{cases}
\end{equation*}
Then we can set a guarantee index $G_t$, to serve as the minimum number of scenarios for which the demand may be met exactly.  We get the constraint
\begin{equation*}
\sum_k x_{kt} \geq G_t, \hspace{3mm} \forall t.
\end{equation*}

The objective function,
\begin{equation*}
%\label{CHEM:objective}
\sum_{t=1}^T\left\{ \expect[r]\left[ \sum_{s=1}^S(\rand{v^s_{skt}} q^{s0}_{skt} - v^b_{st}q^b_{st}) - \sum_{j=1}^J\left(n_{j(t+\delta)} C_{jt} + \sum_{i=1}^I c^o_{ijt}y_{ijt}\right)\right] \right\},
\end{equation*}
is the net present value of the facilities, and includes potential income, capital expenditures and operating expense.  It contains no first stage objective terms, and should obviously be maximized.  Note that $c^o_{ijt} = 0$ if we cannot perform task $i$ with equipment $j$.  Here, $\delta$ is the construction delay once the plant has been ordered.



\subsubsection{Problem statement}
We are given a discrete probability distribution 
\[
\{(P_{kt},\rand{r_{skt}}): k=1,2,\ldots,K\}.
\]
In addition, we require constants $c^o_{ijt}$, $v^b_{st}$, $C_{jt}$, $p_{ij}$, $H_{t}$, $C^o_t$, $f_{si}$, $Q^b_{st}$, $A_{st}^{\text{max}}$, $m_{ij}$, $G_t$, $MC_t$, and $\delta$, and the index sets $I_j^{\text{tasks}}$ and $I_i^{\text{equip}}$.  Then our goal is to\newline
\noindent maximize
\begin{equation*}
\sum_{t=1}^T\left\{ \expect[r]\left[ \sum_{s=1}^S(\rand{v^s_{skt}} q^{s0}_{skt} - v^b_{st}q^b_{st}) - \sum_{j=1}^J\left(n_{j(t+\delta)} C_{jt} + \sum_{i=1}^I c^o_{ijt}y_{ijt}\right)\right] \right\},
\end{equation*}
\noindent subject to
%\end{doublespace}
%\begin{singlespace}
\begin{equation*}
\sum_{i\in I_j^{\text{tasks}}} p_{ij} y_{ijt} \leq H_t N_{jt}, \hspace{3mm} \forall j,t
\end{equation*}
\begin{equation}
\label{CHEM:opex}
\sum_j \sum_{i\in I_j^{\text{tasks}}} c^o_{ijt} y_{ijt}\leq C^o_t \sum_j N_{jt}, \hspace{3mm} \forall t
\end{equation}
\begin{equation}
\label{CHEM:sumbuilt}
N_{jt}= \sum_{\tau = 1}^t n_{j\tau}, \hspace{3mm} \forall j,t
\end{equation}
\begin{equation}
\label{CHEM:matbal2}
A_{st} = A_{s(t-1)} + \sum_i \sum_{j\in I_i^{\text{equip}}} f_{si}B_{ijt} - q^s_{st} + q^b_{st}, \hspace{3mm} \forall s,t
\end{equation}
\begin{equation*}
A_{st} \leq A_{st}^{\text{max}}, \hspace{3mm} \forall s,t
\end{equation*}
\begin{equation*}
B_{ijt} \leq m_{ij} y_{ijt}, \hspace{3mm} \forall i,j,t
\end{equation*}
\begin{equation*}
q^s_{st} = q^{s0}_{skt} + q^{s+}_{skt}, \hspace{3mm} \forall s,k,t
\end{equation*}
\begin{equation*}
q^{s0}_{skt} \leq \rand{Q^s_{skt}}, \hspace{3mm} \forall s,k,t
\end{equation*}
\begin{equation*}
q^b_{st} \leq Q^b_{st}, \hspace{3mm} \forall s,t
\end{equation*}
\begin{equation}
\label{CHEM:caplim}
\sum_{j}C_{jt}n_{jt} + \sum_s v^b_{st} q^b_{st} \leq MC_t, \hspace{3mm} \forall t
\end{equation}
\begin{equation}
\label{CHEM:feasible}
x_{kt}=
	\begin{cases}
		1& \text{if for each $s$, $q^{s0}_{skt} = \rand{Q^s_{skt}}$},\\
		0& \text{otherwise}
	\end{cases}
\end{equation}
\begin{equation}
\label{CHEM:guarantee}
\sum_k x_{kt} \geq G_t, \hspace{3mm} \forall t
\end{equation}
\begin{equation*}
%\label{CHEM:vars}
A_{st}, q^b_{st}, q^s_{st} \geq 0, \hspace{3mm} \forall s,t
\end{equation*}
\begin{equation*}
q^{s0}_{skt}, q^{s+}_{skt} \geq 0, \hspace{3mm} \forall s,k,t
\end{equation*}
\begin{equation*}
 B_{ijt} \geq 0, \hspace{3mm} \forall i,j,t
\end{equation*}
\begin{equation*}
%\label{CHEM:integers}
N_{jt}, n_{jt}, y_{ijt} \in \integers^+.
\end{equation*}
%\end{singlespace}
%\begin{doublespace}

\subsubsection{Numerical results}
Subrahmanyam, Pekny, and Reklaitis \cite{subrahmanyam94} present a problem with $I=4$ tasks, $S=7$ resources, $J=3$ equipment types, $T=2$ time stages, and $K=2$ scenarios per stage.  Operating costs are neglected, so $c_{ijt}^o = 0$ for all cases, and (\ref{CHEM:opex}) is removed from the problem.  Also, the construction delay $\delta$ and constraints (\ref{CHEM:caplim}), (\ref{CHEM:feasible}), and (\ref{CHEM:guarantee}) are not included in the problem.

The parameters for the problem are shown in Tables \ref{CHEM:probabilities}, \ref{CHEM:purchases}, \ref{CHEM:equipment}, and \ref{CHEM:stoich}.  Note that resources $1$ and $2$ must be purchased, resources $4$ and $7$ are sold, and the remainder are intermediate resources.

\begin{table}[ht]
\caption[Probability distribution for test problem in batch chemical plant design]{Probability distribution for random variables (demands not shown are zero)}
\label{CHEM:probabilities}
\[
\begin{array}{|c|c|c|c|c|c|}
	\hline
	kt & P_{kt} & Q^s_{4kt} & Q^s_{7kt} & v^s_{4kt} & v^s_{7kt}\\ \hline
	11 & 0.4    & 0		& 0	    & 51	& 70\\ \hline
	21 & 0.6    & 150	& 200	    & 58	& 80\\ \hline
	12 & 0.4    & 0		& 0	    & 50	& 71\\ \hline
	22 & 0.6    & 150	& 200	    & 59	& 81\\ \hline
\end{array}
\]
\end{table}

\begin{table}[ht]
\caption[Purchase parameters for test problem in batch chemical plant design]{Parameters for purchased resources}
\label{CHEM:purchases}
\[
\begin{array}{|c|c|c|}
	\hline
	st	& v^b_{st}	& Q^b_{st}\\ \hline
	11	& 23		& 200\\ \hline
	12	& 24		& 200\\ \hline
	21	& 25		& 250\\ \hline
	22	& 26		& 250\\ \hline
\end{array}
\]
\end{table}

\begin{table}
\caption[Equipment parameters for test problem in batch chemical plant design]{Parameters for equipment types}
\label{CHEM:equipment}
\[
\begin{array}{|c|c|c|c|}
	\hline
	j	& C_{j1}	& C_{j2}	& m_{ij} \forall i\\ \hline
	1	& 2500		& 2600		& 100\\ \hline
	2	& 3000		& 3100		& 200\\ \hline
	3	& 2800		& 2900		& 150\\ \hline
\end{array}
\]
\end{table}

\begin{table}
\caption[Stoichiometric coefficients for test problem in batch chemical plant design]{Stoichiometric coefficients $f_{si}$}
\label{CHEM:stoich}
\[
\begin{array}{|c|c|c|c|c|c|c|c|}
\hline
i \backslash s	& 1	& 2	& 3	& 4	& 5	& 6	& 7\\ \hline
1		& -1	& -1	& 1	& 0 	& 0	& 0 	& 0\\ \hline
2		& 0	& 0	& -1	& 1	& 1	& 0	& 0\\ \hline
3		& 0	& -1	& 0	& 0	& -1	& 1	& 0\\ \hline
4		& 0	& 0	& 0	& 0	& 0	& -1	& 1\\ \hline
\end{array}
\]
\end{table}

Additionally, 
\[
A_{st}^{\text{max}} = 400 \hspace{3mm} \forall s,t,
\]
\[
H_1 = H_2 = 80 \text{ days},
\]
\[
p_{11} = p_{41} = p_{12} = p_{42} = p_{23} = p_{33} = 4,
\] 
\[
I_1^{\text{tasks}} = I_2^{\text{tasks}} = \{1,4\},
\]
and
\[
I_3^{\text{tasks}} = \{2,3\}.
\]


The optimal objective value stated in \cite{subrahmanyam94} is $3300$, with optimal values of $N_{21} = 1$ and $N_{31}=1$.

The problem \url{chem} in our collection is an attempt at recreating this example.  We have not been able to verify that we have succeeded in this attempt, as we have only run \url{chem} as a continuous model.  The optimal objective value for \url{chem} as a continuous model is $13009.16667$.  


\subsubsection{Notational reconciliation}

For simplicity, equations (\ref{CHEM:feasible}) and (\ref{CHEM:guarantee}) are removed from the problem, and $\delta$ is set to $0$.  The transition of notation is then quite straightforward.  Define the diagonal matrix $\Delta_j^{\text{tasks}} \in \reals^{I \times I}$ by
\[
(\Delta_j^{\text{tasks}})_ii \assign 
\begin{cases}
	1	&	\text{if $i \in I_j^{\text{tasks}}$}\\
	0	&	\text{otherwise.}
\end{cases}
\]

Then, we make the following definitions, and also introduce slack variables $u^1, u^2, \ldots, u^7$:
{{\allowdisplaybreaks{\begin{gather*}
p_j \assign \left[
\begin{array}{c}
	p_{1j}\\
	\vdots\\
	p_{Ij}
\end{array}
\right];
\hspace{0.2in}
p \assign \left[
\begin{array}{cccc}
	p_1 & & & 0\\
    & p_2 & & \\
	& & \ddots &\\
	0 & & &p_J
\end{array}
\right];\\
y_{jt} \assign \left[
\begin{array}{c}
	y_{1jt}\\
	\vdots\\
	y_{Ijt}
\end{array}
\right];
\hspace{0.2in}
y_t \assign \left[
\begin{array}{c}
	y_{1t}\\
	\vdots\\
	y_{Jt}
\end{array}
\right];
\hspace{0.2in}
B_{jt} \assign \left[
\begin{array}{c}
	B_{1jt}\\
	\vdots\\
	B_{Ijt}
\end{array}
\right];\\
B_t \assign \left[
\begin{array}{c}
	B_{1t}\\
	\vdots\\
	B_{Jt}
\end{array}
\right];
\hspace{0.2in}
N_t \assign \left[
\begin{array}{c}
	N_{1t}\\
	\vdots\\
	N_{Jt}
\end{array}
\right];
\hspace{0.2in}
n_t \assign \left[
\begin{array}{c}
	n_{1t}\\
	\vdots\\
	n_{Jt}
\end{array}
\right];\\
u^i_t \assign \left[
\begin{array}{c}
	u^i_{1t}\\
	\vdots\\
	u^i_{St}
\end{array}
\right], i=3,5,6;
\hspace{0.2in}
u^1_t \assign \left[
\begin{array}{c}
	u^1_{1t}\\
	\vdots\\
	u^1_{Jt}
\end{array}
\right];\\
u^4_t \assign \left[
\begin{array}{c}
	u^4_{11t}\\
	u^4_{12t}\\
	\vdots\\
	u^4_{IJt}
\end{array}
\right];
\hspace{0.2in}
c^o_{jt}\assign \left[
\begin{array}{c}
	c^o_{1jt}\\
	\vdots\\
	c^o_{Ijt}
\end{array}
\right];
\hspace{0.2in}
c^o_t \assign \left[
\begin{array}{c}
	c^o_{1t}\\
	\vdots\\
	c^o_{Jt}
\end{array}
\right];\\
C_t \assign \left[
\begin{array}{c}
	C_{1t}\\
	\vdots\\
	C_{Jt}
\end{array}
\right];
\hspace{0.2in}
\hat{A}_t \assign \left[
\begin{array}{c}
	A_{1t}\\
	\vdots\\
	A_{St}
\end{array}
\right];
\hspace{0.2in}
A_t^{\text{max}} \assign \left[
\begin{array}{c}
	A^{\text{max}}_{1t}\\
	\vdots\\
	A^{\text{max}}_{St}
\end{array}
\right];\\
\rand{Q^s_t} \assign \left[
\begin{array}{c}
	\rand{Q^s_{1kt}}\\
	\vdots\\
	\rand{Q^s_{Skt}}
\end{array}
\right];
\hspace{0.2in}
Q^b_t \assign\left[
\begin{array}{c}
	Q^b_{1t}\\
	\vdots\\
	Q^b_{St}
\end{array}
\right];
\hspace{0.2in}
v^b_t \assign \left[
\begin{array}{c}
	v^b_{1t}\\
	\vdots\\
	v^b_{St}
\end{array}
\right];\\
\rand{v^s_t} \assign \left[
\begin{array}{c}
	\rand{v^s_{1kt}}\\
	\vdots\\
	\rand{v^s_{Skt}}
\end{array}
\right];
\hspace{0.2in}
q^s_t \assign \left[
\begin{array}{c}
	q^s_{1t}\\
	\vdots\\
	q^s_{St}
\end{array}
\right];
\hspace{0.2in}
q^{s0}_t \assign \left[
\begin{array}{c}
	q^{s0}_{1kt}\\
	\vdots\\
	q^{s0}_{Skt}
\end{array}
\right];\\
q^{s+}_t \assign \left[
\begin{array}{c}
	q^{s+}_{1kt}\\
	\vdots\\
	q^{s+}_{Skt}
\end{array}
\right];
\hspace{0.2in}
M \assign \left[
\begin{array}{cccc}
	m_{11}	&	&	& 0\\
	&	 m_{21}	&	& \\
	&		&\ddots	& \\
	0	&	&	& m_{IJ}
\end{array}
\right];\\ 
f_s \assign \left[
\begin{array}{c}
	f_{s1}\\
	\vdots\\
	f_{sI}
\end{array}
\right];
\hspace{0.2in}
F_s \assign f_s\trp \left[
\begin{array}{cccc}
	\Delta_1^{\text{tasks}} & \Delta_2^{\text{tasks}} & \cdots & \Delta_J^{\text{tasks}}
\end{array}
\right]\\
F \assign \left[
\begin{array}{c}
	F_1\\
	\vdots\\
	F_S
\end{array}
\right];
\hspace{0.2in}
\Delta^{\text{tasks}} \assign \left[
\begin{array}{cccc}
	\Delta_1^{\text{tasks}} & & & 0\\
    & \Delta_2^{\text{tasks}} & & \\
	& & \ddots &\\
	0 & & &\Delta_J^{\text{tasks}}
\end{array}
\right].
\end{gather*}
}}}

Then, let 
\begin{equation}
\label{CHEM:xt}
x_t\assign\left[
\begin{array}{c}
	y_t\\
	N_t\\
	n_t\\
	\hat{A}_t\\
	B_t\\
	q^s_t\\
	q^b_t\\
	q^{s0}_t\\
	q^{s+}_t\\
	u^1_t\\
	u^2_t\\
	u^3_t\\
	u^4_t\\
	u^5_t\\
	u^6_t\\
	u^7_t
\end{array}
\right],
\hspace{0.2in}
\hbox{and }
\hspace{0.2in}
c_t\assign \left[
\begin{array}{c}
	-c^o_t\\
	0^{J\times 1}\\
	-C_t\\
	0^{S\times 1}\\
	0^{IJ\times 1}\\
	0^{S\times 1}\\
	-v^b_t\\
	\rand{v^s_t}\\
	0^{S\times 1}\\
	0^{J\times 1}\\
	0\\
	0^{S\times 1}\\
	0^{IJ\times 1}\\
	0^{S\times 1}\\
	0^{S\times 1}\\
	0
\end{array}
\right].
\end{equation}

We assign $A_t$ according to Figure \ref{CHEM:at}.  The vertical lines separate parts of the matrix according to (\ref{CHEM:xt}).  So, for example, the second partition of the matrix corresponds will be multiplied by $N_t$.  All blanks are zero.  Note that the double sum in equation (\ref{CHEM:matbal2}) may be expressed as
\[
\sum_j \sum_{i\in I_j^{\text{tasks}}} f_{si} B_{ijt}.
\]

The transition matrix $T_t$ is defined in Figure \ref{CHEM:tt}.  There are only two nonzero entries.  One is from (\ref{CHEM:matbal2}), and the other is from (\ref{CHEM:sumbuilt}), which may be rewritten
\[
N_{jt} = N_{jt-1} + n_{jt}.
\]

Our transition to the notation of Problem (\ref{PROB:mslp}) is complete if we set
\begin{equation*}
\rand{b_t} \assign \left[
\begin{array}{c}
	0^{J\times J}\\
	0\\
	0^{J\times J}\\
	0^{S\times S}\\
	A^{\text{max}}_t\\
	0^{IJ\times IJ}\\
	0^{S\times S}\\
	\rand{Q^s_t}\\
	Q^b_t\\
	MC_t
\end{array}
\right].
\end{equation*}



\begin{landscape}
\begin{figure}[ht]
\caption{Array $A_t$ for chemical design planning example}
\label{CHEM:at}
\small{
\begin{multline*}
A_t\assign\left[
\begin{array}{c|c|c|c|c|c|c|c|c|}%c|c|c|c|c|c|c
	p\trp \Delta^{\text{tasks}}&-H_t I^{J\times J}&&&&&&&%&%I^{J\times J}&&&&&&
\\ \hline
	(c^o_t)\trp \Delta^{\text{tasks}}&-C^o_t 1^{1\times J}&&&&&&&%&&1&&&&&
\\ \hline
	&I^{J\times J}&-I^{J\times J}&&&&&&%&%&&&&&&
\\ \hline
	&&&I^{S\times S}&-F&I^{S\times S}&-I^{S\times S}&&%&%&&&&&&
\\ \hline
	&&&I^{S\times S}&&&&&%&%&&I^{S\times S}&&&&
\\ \hline
	-M&&&&I^{IJ\times IJ}&&&&%&%&&&I^{IJ\times IJ}&&&
\\ \hline
	&&&&&I^{S\times S}&&-I^{S\times S}&-I^{S\times S}%&%&&&&&&
\\ \hline
	&&&&&&&I^{S\times S}&%&%&&&&I^{S\times S}&&
\\ \hline
	&&&&&&I^{S\times S}&&%&%&&&&&I^{S\times S}&
\\ \hline
	&&C_t\trp&&&&(v_t^b)\trp&&%&%&&&&&&1
\end{array}
\right.\\[0.3in]
\left.
\begin{array}{c|c|c|c|c|c|c}
	I^{J\times J}&&&&&&\\ \hline
	&1&&&&&\\ \hline
	&&&&&&\\ \hline
	&&&&&&\\ \hline
	&&I^{S\times S}&&&&\\ \hline
	&&&I^{IJ\times IJ}&&&\\ \hline
	&&&&&&\\ \hline
	&&&&I^{S\times S}&&\\ \hline
	&&&&&I^{S\times S}&\\ \hline
	&&&&&&1
\end{array}
\right]\\
\end{multline*}
}
\end{figure}
\end{landscape}

%\begin{landscape}
\begin{figure}[ht]
\caption{Array $T_t$ for chemical design planning example}
\label{CHEM:tt}
%\small{
\begin{multline*}
T_t\assign\left[
\begin{array}{c|c|c|c|c|c|c|c|c|c|c|c|c|c|c|c|c}
	&&&&&&&&&&&&&&&&\\ \hline
	&&&&&&&&&&&&&&&&\\ \hline
	&-I^{J\times J}&&&&&&&&&&&&&&&\\ \hline
	&&&-I^{S\times S}&&&&&&&&&&&&&\\ \hline
	&&&&&&&&&&&&&&&&\\ \hline
	&&&&&&&&&&&&&&&&\\ \hline
	&&&&&&&&&&&&&&&&\\ \hline
	&&&&&&&&&&&&&&&&\\ \hline
	&&&&&&&&&&&&&&&&\\ \hline
	&&&&&&&&&&&&&&&&
\end{array}
\right]\\
\end{multline*}
%}
\end{figure}%
